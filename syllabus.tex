% Fall 2025 NLP

\documentclass[12pt]{article}
\setlength{\parindent}{0pt}
\setlength{\baselineskip}{1.5pt}
\setlength{\parskip}{6pt}

\pagestyle{plain} 
\addtolength{\textwidth}{.8in}
%\addtolength{\hoffset}{-.4in}
%\addtolength{\voffset}{-1.0in}

\usepackage{multicol}    % For multiple columns.
\usepackage{paralist}
\usepackage{footnote}
\makesavenoteenv{tabular}
\makesavenoteenv{table}
\makesavenoteenv{center}
\usepackage{enumitem}
\usepackage{amsmath}     % For dfrac.
\usepackage[hyperfootnotes=false]{hyperref}    % For URLs and hyperlinks.
\usepackage{enumerate}   % For custom enumeration annotations.
\usepackage{xcolor}       % For colors.
\usepackage{graphicx}    % For figures.
\usepackage{geometry}    % For figures.
\usepackage[normalem]{ulem}    % For strikeouts.

\renewcommand*{\thefootnote}{\fnsymbol{footnote}}
\newcommand{\freakingtilde}{\raisebox{0.5ex}{\texttildelow}}

   
\usepackage{color, colortbl}
\definecolor{Gray}{gray}{0.9}

\begin{document}

\begin{center}
\begin{Large}
\textbf{DATA 470D3 --- Natural Language Processing} \\
\end{Large}
\vspace{.10in}
\begin{small}
Professor: Stephen Davies\\
Fall semester 2025\\
\vspace{.05in}
Class: TR 3:30pm in Farmer 054\\
\vspace{.05in}
Final exam: Thursday, Dec 11th, 3:30--6pm\\
\vspace{.10in}
\end{small}
\input{../office}

\vspace{.10in}

\textbf{\url{http://stephendavies.org/nlp}}

\end{center}

Much of the data we work with in Data Science can be considered
\textbf{structured} data. It's rigidly unambiguous, and conforms wholly to some
``schema'' that defines the various parts it contains and what they mean. Most
structured data is pretty easy to work with, and that's because its authors
made it so. After all, the main point of collecting information in (say) a
tabular format is so that it can be consulted and queried easily.

In stark contrast to this is \textbf{unstructured data}, which is any
electronic material that does \textit{not} conform to a schema. Images and
sound files are in this category, as are videos, but the most common is
\textbf{text} expressed in a \textbf{natural language} like Spanish, Chinese,
or English. Some experts estimate that over 80\% of the data used by any
organization consists of unstructured text. What a treasure trove, if we can
make use of it!

The field of NLP, and this class, will be focused on extracting meaning from
text despite its lack of structure. We'll learn the linguistic concepts and the
statistical tools necessary to approach this maddeningly-inconsistent domain,
and write programs that can derive insight from something as formal as a
journal article or as informal as a tweet.

As even your grandparents know, NLP applications have positively exploded in
reach and popularity over the last several years, including the debut of
ChatGPT to the public in November 2022. I consider this quite possibly the most
important technology of our lifetime -- perhaps ever -- and breaking it down
to first principles will be a highlight of this semester. Without being too
egocentric, let me suggest there's a decent chance that this will be the most
important college course you ever take.

\section*{Course Objectives}

\begin{itemize}
\small

\item To survey the field of Natural Language Processing (NLP), so that you
know what tasks it can currently accomplish and what is still on the horizon.

\item To experiment with pre-neural NLP algorithms, in order to appreciate
their capabilities and limitations.

\item To examine in detail a modern neural architecture and illuminate how it
achieves its magic.

\item To give you the opportunity to write real programs running on real text
data sets, and to see what kinds of practical issues are involved.

\item To confront some of the major ethical concerns that NLP involves,
including privacy and consent; bias and fairness; transparency and
accountability; and intellectual property and ownership.

\end{itemize}

\section*{Student Learning Outcomes}

After completing this course, students will be able to...

\begin{itemize}

\item ...identify the different tasks associated with the field of Natural
Language Processing (NLP), and to explain their purposes, assumptions,
promises, and limits.

\item ...create custom NLP models in PyTorch for small corpora, and evaluate
their effectiveness.

\item ...obtain (\textit{e.g.}, from HuggingFace) and fine-tune existing NLP
models for small corpora, and evaluate their effectiveness.

\item ...make use of some key NLP-related Python libraries and apply them to
larger corpora.

\item ...both:

    \begin{compactitem}
        \item comprehend the statistical theory underpinning important NLP
        algorithms, and
        \item apply that theory and implement those algorithms concretely in a
        programming language (Python).
    \end{compactitem}

\item ...confront and make informed decisions regarding some of the major
ethical concerns that NLP involves, including privacy and consent; bias and
fairness; transparency and accountability; and intellectual property and
ownership.

\end{itemize}

\section*{Rules of the game}

\begin{compactenum}

\item There are absolutely, positively, NO stupid questions!! Your job is
not to already know everything before you start the course. Your job is to
try hard to learn, and part of that involves asking questions. I'm a nice
guy, and I will not ever belittle you, snub you, or make fun of you; and if
anyone else does so I will personally break both of their arms.

\item This class will be interactive. When I point at you in class, say
your first name, and be prepared to try and answer questions. (Don't worry
if you don't know all the answers.) 

\item The books we'll be using (my \textit{Quick Steep Climb} text, and
Jurafsky \& Martin's \textit{Speech and Language Processing}) are mandatory,
and you will be required to actually read the sections I assign. Luckily for
you, J\&M is (1) an extremely well-written book, and (2) all
\href{https://web.stanford.edu/~jurafsky/slp3/}{freely available online} thanks
to Dan and Jim. \textit{QSC} is also
\href{http://stephendavies.org/quick.pdf}{free and open source}.

\item The reading checks, quizzes and the final exam will cover both (1)
lecture material from class, and (2) the assigned readings. Much of what's in
\#2 will make an appearance in \#1, but this is not guaranteed.

\item Don't skip class. Just don't. It's bad form. I work hard to prepare for
class, to make it compelling and relevant. It hurts my feelings when you don't
come. Plus you miss out on important stuff, and you'll end up falling behind
if you skip lecture.  So come every time. Come happy, fresh, excited, ready to
think and to participate. 

\item \textbf{Absolutely no laptops, cell phones, or other devices during
class.} I've had students claim that they take notes on their laptop during
lecture, but even if it's true, those things are way too big a distraction to
you and your fellow students to make it worth it. Just stay tuned in, because I
move fast.

\item For any Zoom class that may take place this semester, \textbf{you must
have your webcam on during the entirety of the lecture.} If you don't have a
working Webcam, buy one immediately.

\end{compactenum}

\section*{Books}

\begin{itemize}

\item \textbf{QSC:} \textit{A Quick, Steep Climb Up Linear Algebra}, version
1.1.0: Davies, Stephen. 2021. Available online at
\url{http://stephendavies.org/quick.pdf}.

\item \textbf{J\&M}: \textit{Speech and Language Processing}, 3rd Edition
Draft, Jan 12 2025 release: Jurafsky, Daniel and Martin, James. 2025. Available
online at \url{https://web.stanford.edu/~jurafsky/slp3/}.

\end{itemize}

I'll be surgically assigning specific sections of QSC to build the background
you need to work with vectors, matrices, and tensors in PyTorch. If you're a
Computer Science major and have already taken CPSC 284, this will be review for
you. But I'm warning you: this stuff needs to be \textit{sharp} in your mind.
Do actually plan on spending time reviewing.

J\&M is still unfinished (Jim told me via email that the work on the 3rd
Edition is proceeding ``at a glacial pace'') and tbh probably will always be.
But it's hands-down one of the best technical books I've ever read. The authors
are giants in the field -- legends, even -- and they know NLP as well as anyone
on the planet ever has. Enjoy their freely available book!

\section*{The Honor Code and this course}

For this course:

\begin{itemize}
\itemsep.1em

\item The linear algebra reading checks will take place in class, at the very
start of the class period the first three weeks. They are closed-book,
closed-notes, and timed (at about 10 minutes).

\item The Canvas quizzes are open-book and open-notes, but they must be taken
alone, in a quiet place, without any form of contact with anyone, and
\textit{without any use of any website} other than the book's website and the
class website. \textit{No} ChatGPT or other form of AI is allowed. These
quizzes will be timed generously at anywhere from 30-60 minutes.

\item \textbf{You must not use any part of any other student's Python code.}
You may not directly copy any part of anyone else's program. That being said,
it is okay to work alongside classmates in a study group, and you are allowed
to talk through homework assignments, ask others how they solved a particular
problem, and so forth. \textbf{If you do this, you must name all the people you
worked with on your homework submission.} If you fail to give this information
clearly on your homework submission, it is an Honor Code violation.

\item You \textit{are} permitted to use AI (such as ChatGPT) while working on
the homeworks, but \textbf{in your submission you \textit{must} include link(s)
to all the chat(s) you created while working on it.} If you fail to give this
information clearly in your homework submission, it is an Honor Code
violation.

\end{itemize}


\section*{Late policy}

No late work will be accepted this semester. Get your stuff in on time!

\bigskip

\section*{Grading}

Grading this semester will be based on ``experience points" (XP). As you
complete activities, you will earn XP towards your final total. XP can never
be lost, only gained, but you have to earn what you get (\textit{i.e.}, you
don't ``start off with a 100\%" and lose points based on mistakes you make).

There will be opportunities to earn XP throughout the course. Some of these
will be spontaneous as the mood strikes me. Some you can earn by completing
in-class activities. Some may be in response to impressive things I see you
do as the semester progresses. The following opportunities are
\textit{guaranteed} to be available to you:

\begin{center}
\small
\textbf{Guaranteed XP opportunities:}\\
\begin{tabular}{|l | c | c|}
\hline
Activity & Possible XP \\
\hline
\hline
Five linear algebra reading checks     & 5 each  \\
\hline
Six open-book, open-note, timed Canvas quizzes     & 25 each  \\
\hline
Five homework assignments     & 40 each  \\
\hline
2-page paper: ethical considerations of LLMs       & 50 \\
\hline
Final exam (comprehensive)    & 150 \\
\hline
Various and sundry others     & varies \\ 
\hline
\end{tabular}
\end{center}
% Total: 575

\subsection*{Grading levels}
Here are the levels you may achieve, together with the grade awarded (if any)
and the points necessary to reach!

\footnotesize
\begin{center}
\begin{tabular}{|l|c|c|c|c|c|}
\hline
Level & \textbf{Total XP} & Semester grade \\
\hline
William Shakespeare    & \textbf{550} & A+ \\
Homer                  & \textbf{500} & A \\
Fyodor Dostoevsky      & \textbf{450} & A-- \\
Charlotte Br\"{o}nte   & \textbf{400} & B+ \\
Leo Tolstoy            & \textbf{375} & B \\
Sophocles              & \textbf{350} & B-- \\
Jane Austen            & \textbf{330} & C+ \\
Toni Morrison          & \textbf{310} & C \\
Emily Dickinson        & \textbf{290} & C-- \\
Geoffrey Chaucer       & \textbf{270} & D+ \\
Charles Dickens        & \textbf{250} & D \\
George Eliot           & \textbf{230} & D-- \\
Isaac Asimov           & \textbf{210} & \\
C.S. Lewis             & \textbf{190} &  \\
George Orwell          & \textbf{170} &  \\
Virginia Woolf         & \textbf{150} &  \\
Ursula Le Guin         & \textbf{130} &  \\
J.R.R. Tolkien         & \textbf{110} & \\
Mary Shelley           & \textbf{90} & \\
Mark Twain             & \textbf{75} & \\
Lewis Carroll          & \textbf{60} &  \\
Agatha Christie        & \textbf{45} &  \\
Sir Arthur Conan Doyle & \textbf{30} &  \\
E.L. James             & \textbf{20} &  \\
Edward Bulwer-Lytton   & \textbf{10} &  \\
Zerna Addis Sharp      & \textbf{0} & \\
\hline
\end{tabular}
\end{center}
\normalsize

\section*{Submitting homeworks}

Rules for submitting homeworks (whether in Python or in English) will be
given when the homework is assigned. For programs, you'll be emailing me your
program code as an attachment, and using a specific subject line to
distinguish it from my hordes of other email. Meeting the deadline is a
matter of sending your email before time expires. For responsive readings,
\textbf{you must submit a hardcopy of your paper -- nothing electronic will
serve as an adequate substitute.} I'll probably collect these in class.

Also, most of my homeworks are due at ``midnight." Here's what ``midnight"
means: if a homework is due ``at midnight on Thursday," then it is due after
all of Thursday has elapsed, and the clock strikes twelve. (In other words,
this is good news: you have all Thursday to work on it.)

\section*{Basis for determining mid-semester reports}

For midterm progress reports, I look mostly at (a) whether you've been
turning assignments in (and on time), and (b) quiz scores. If either or both
of these categories are lacking, it's a sign of danger, and I will give you a
``U" for your mid-semester grade. Please don't hesitate at all to come talk
to me about this so we can figure out how you can do better in the course.


\section*{Use of Artificial Intelligence (AI) technologies}

AI is \textbf{not} permitted on any quiz or exam. Doing so will be considered a
violation of course policy and as such, the student may be referred to the UMW
Honor Council for a violation of academic integrity.

You \textit{are} permitted to use AI (such as ChatGPT) while working on the
homeworks, but \textbf{in your submission you \textit{must} include link(s) to
all the chat(s) you created while working on it.} Failure to do so will be
considered a violation of course policy and as such, the student may be
referred to the UMW Honor Council for a violation of academic integrity.

Similarly, you \textit{are} permitted to use AI (such as ChatGPT) while
composing your responsive readings, but \textbf{in your submission you
\textit{must} include link(s) to all the chat(s) you created while working on
it.} Failure to do so will be considered a violation of course policy and as
such, the student may be referred to the UMW Honor Council for a violation of
academic integrity.

\input{../standard}

\section*{How to reach you}

I will be communicating with you outside of class time via e-mail, so make
sure to check your UMW e-mail every day! I will also post announcements to
the course website, so be sure to subscribe to its RSS feed and check it in
your feed reader at least once a day!

\section*{Calendar}

The official calendar for the course, complete with assignment due dates,
tests, \textit{etc.}, will be maintained on the course website at
\url{http://stephendavies.org/nlp}. \textbf{In any way that the website
conflicts with the tentative calendar below, the website is to be considered
correct, and the tentative calendar below out of date.}

\addtolength{\hoffset}{-.4in}
\begin{center}
\begin{tabular}{|c|l|c|}
\hline
Week & Topics & Due \\
% Note: week 8 is spring break
% Also ch. 12?

% HW0 - corpus
% HW? - regexs Emily Dickinson
% HW1 - BPE algorithm
% HW2 - bigram LM for your corpus
% HW3 - Torch practice, plus compute PPMI for your corpus
% HW4 - fine-tune an LM on your corpus for text generation?
% HW5 - fine-tune an LM on your corpus and a friend's for text classification?
\hline
1  & Regex's, corpora, words and tokens, tokenization & RC's \#1\&2 \\
2  & N-gram models, BoW models, text classification   & RC's \#3\&4 \\
3  & PyTorch tensor calculation, logistic regression  & RC \#5, HW \#0 \\
4  & Calc primer; stochastic gradient descent         & HW \#1 \\
5  & Vector semantics and embeddings                  & Quiz 1\\
6  & Neural models, in theory and in PyTorch          & HW \#2 \\
7  & Training neural models: backprop and autodiff    & Quiz 2\\
8  & \textit{\color{gray} (fall break)} \ The attention mechanism & HW \#3 \\
9  & The transformer architecture                     & Quiz 3 \\
10 & Contextual and positional embeddings             & Quiz 4\\
11 & The HuggingFace code base, hub, and community    & HW \#4 \\
12 & Training, sampling, and evaluating LLMs          & Quiz 5 \\
13 & Machine translation and summarization            & Quiz 6 \\
14 & Question answering and RAG \ \textit{\color{gray} (Thanksgiving)} & HW \#5 \\
15 & Dialogue systems and future trends in NLP        & Paper \\
Finals &                                              & Final exam \\
\hline
\end{tabular}
\end{center}

\end{document}
